\documentclass[a4paper,10pt]{article}
\usepackage[utf8]{inputenc}
\usepackage{polski}

\title{Zwiększanie rozdzielczości bitmapy}
\author{Sławomir Górawski\\ Aleksander Czeszejko-Sochacki}

\begin{document}
 \maketitle
 \section{Wstęp}
 Zarówno podczas powiększania obrazu, jak i zmiany jego
 rozdzielczości pojawia się następujący problem: jaki odcień 
 powinny mieć nowe piksele? Muszą być one, rzecz jasna, związane
 z odcieniami pikseli z obrazu przed przetworzeniem. Do ich
 ustalenia stosuje się rozmaite metody numeryczne. Pewnym z nich
 przyjrzymy się poniżej.
 
 \section{Cel wykonanych doświadczeń}
 Aby móc określić, który sposób przekształcania obrazu jest
 najlepszy, musielibyśmy znać pewne wskaźniki jakości zmiany
 rozdzielczości. Zamiast tego przetestujemy trzy metody
 i ocenimy je dwojako:
 \begin{itemize}
  \item Porównamy wizualnie, która metoda najlepiej
  odzwieciedla obraz sprzed przekształcenia. Porównamy ostrość
  krawędzi i stopień rozmycia obrazu.
  \item Porównamy czasy działania wszystkich metod.
  \item Sprawdzimy, czy kolejność skalowania (czy najpierw
  w pionie, czy w poziomie) ma znaczenie.
 \end{itemize}
 
 \section{Opis użytych metod}
  \begin{center}
   \textbf{Dane: bitmapa o 256 odcieniach szarości o rozmiarze $N_x\times N_y$\\
   Szukane: bitmapa o 256 odcieniach szarości o rozmiarze $M_x\times M_y$}\\
  \end{center}
  
  przy czym $N_x < M_x$ i $N_y < M_y$.\\
  
  Przyjmijmy, że rozszerzanie obrazu odbywa się niezależnie
  w pionie i w poziomie, to znaczy najpierw rozszerzamy obraz
  w jednym, potem w drugim kierunku. Dodatkowo, użyte metody
  ograniczają się do rozszerzania obrazów o rozmiarze Mx1, tzn.
  wektora $[p_1, p_2, \dots, p_M]$. Cały wejściowy obraz można
  potraktować jako sumę takich wektorów i każdy z nich 
  niezależnie przeskalować.
  
  \newpage
  Poniżej przyjmujemy $K(p_i)$ - wartość odcienia
  w punkcie $p_i$.
  
  Użyte metody:
  \begin{enumerate}
   \item \textit{metoda najbliższego sąsiada}, wartość odcienia
   piksela $p_i$:
   \[K(p_i) = K(round(p_i)) (i = 1, 2,\dots, N)\]
   co oznacza, że każdemu nowemu pikselowi przyporządkowujemy
   odcień starego piksela o najbliższym indeksie.
   
   \item \textit{metoda bazująca na funkcji sklejanej pierwszego
   stopnia}, konstruujemy taką funkcję sklejaną pierwszego stopnia
   S, że:
   \[S(t_i) = K(t_i) (i = 1, 2,\dots, M)\]
   a następnie przyjmujemy:
   \[K(p_i) = S(p_i)\]
   
   \item \textit{metoda bazująca na naturalnej funkcji sklejanej
   trzeciego stopnia}, konstruujemy taką naturalną funkcję sklejaną
   trzeciego stopnia Z, że:
   \[Z(t_i) = K(t_i) (i = 1, 2,\dots, M)\]
   a następnie przyjmujemy:
   \[K(p_i) = Z(p_i)\]
  \end{enumerate}
  
  Wszystkie wymienione metody zostały zaimplementowane w pliku
  \texttt{program.jl}, w następujących funkcjach: 
  \texttt{expand\_using\_nearest\_neighbours, 
  expand\_using\_linear\_spline,
  expand\_using\_cubic\_spline}\\
  
  \section{Testy}
  
  W pliku \texttt{program.ipynb} znajduja się obrazy, dla których
  powyższe metody zostały przetestowane:
  
  \begin{center}
   \begin{tabular}{| l | l | l |}
    \hline
     Nazwa                                          &  Rozmiar domyślny  &  Rozmiar po zmianie rozdzielczości\\
     \textquotedblleft Cameraman\textquotedblright  &  $200\times 200$   &  $400\times 400$\\ 
     \hline
    \end{tabular}
   \end{center}

\end{document}
